\documentclass{beamer}
\usepackage[T2A]{fontenc}
\usepackage{ dsfont }
\DeclareSymbolFont{T2Aletters}{T2A}{cmr}{m}{it}
\usepackage[utf8]{inputenc}
\setbeamertemplate{footline}[frame number]
\usetheme{Pittsburgh}
\usecolortheme{seahorse}
\setbeamertemplate{items}[default]
\title[Bayecho]{BayesHammer - Hammer + Echo = Bayecho\\Скрещивая подходы}
\author{Дмитрий Грошев}
\date{24.03.2012}
\institute{}

\begin{document}

\begin{frame}
\titlepage
\end{frame}

\begin{frame}{О жизни}
  Все делают ошибки:
  \begin{itemize}
  \item Бог делает ошибки
  \item Человек делает ошибки
  \item Секвенатор делает ошибки
  \end{itemize}
  Однако, последние можно попытаться исправить
\end{frame}

\begin{frame}{Об ошибках}
  Используя некоторые наблюдения, можно попытаться исправить ошибки в ридах
  \begin{itemize}
  \item Геном достаточно неравномерен, чтобы риды в большинстве случаев сильно отличались друг от друга
  \item Покрытие каждого нуклеотида значительно больше 1
  \item Секвенаторы Illumina редко делают инделы, чаще замены
  \item Секвенаторы Illumina делают больше ошибок ближе к концу рида
  \item Ошибки обычно независимы друг от друга и зависят от исходного нуклеотида
  \end{itemize}
\end{frame}

\begin{frame}{Hammer}
  \begin{itemize}
  \item k-меры кластеризуются по дистанции Хемминга (k $\approx$ 55), кластер = группа связанности в графе Хемминга
  \item В полученных кластерах ищется консенсус
  \item Полученные k-меры считаются верными
  \end{itemize}
\end{frame}

\begin{frame}{BayesHammer}
  \begin{itemize}
  \item Так же находим кластеры k-меров
  \item Кластеризуем сами кластеры ещё раз, анализируя вероятности замены из q-values
  \item Находим центры кластеров — верные k-меры
  \item Расширяем множество верных k-меров — рид, целиком покрытый верными k-мерами, целиком верен
  \item Исправляем риды, внося изменения из исправленных k-меров
  \end{itemize}
\end{frame}

\begin{frame}{Echo}
  \begin{itemize}
  \item Анализируются риды, а не k-меры
  \item Ищутся достаточно сильно перекрывающиеся риды
  \item Вводится понятие confusion matrix
    \begin{equation*}
      \Phi^{(m)}_{b,b'} = \mathds{P}(r_m = b' | H_m = b)
    \end{equation*}
    $r_m$ — нуклеотид в риде в позиции m\\
    $H_m$ — нуклеотид в истинном сиквенсе в позиции m
  \item Ищутся наиболее вероятные нуклеотиды в перекрывающихся областях
  \end{itemize}
\end{frame}

\begin{frame}{Bayecho}
  Почему бы не взять лучшее от обоих?
  \begin{itemize}
  \item Анализировать риды, а не k-меры
  \item Кластеризовать перекрывающиеся области ридов, используя confusion matrix
  \item Возможно, кластеризация перекрытий позволит уменьшить влияние ошибок на стадии отбора пересечений
  \end{itemize}
\end{frame}

\begin{frame}{ }
  \begin{center}
    WUT?
  \end{center}
\end{frame}

\end{document}
